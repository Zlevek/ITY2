\documentclass[11pt, a4paper, twocolumn]{article}
\usepackage[text={18cm,25cm}, top=2.5cm, left=1.5cm]{geometry}
\usepackage[utf8]{inputenc}
\usepackage[IL2]{fontenc}
\usepackage[czech]{babel}
\usepackage[unicode]{hyperref}
\usepackage{times}
\usepackage{amsthm}
\usepackage{amsmath} %normalni text v mathmode
\usepackage{amssymb} %symbol racionalnich cisel

\newtheorem{definition}{Definice}
\newtheorem{lemma}{Věta}

\begin{document}
\begin{titlepage}
\begin{center}
\Huge
\textsc{Fakulta informačních technologií\\
        Vysoké učení technické v Brně}\\
\vspace{\stretch{0.382}}
{\LARGE Typografie a publikování\,--\,2. projekt}
\vspace{0.3 em}
{\LARGE Sazba dokumentů a matematických výrazů\\}
\vspace{\stretch{0.618}}
\end{center}
\lineskip=1pt
{\large 2019 \hfill Martina Zlevorová (xzlevo00)}
\end{titlepage}

\section*{Úvod}
V této úloze si vyzkoušíme sazbu titulní strany, matematických vzorců, prostředí a dalších textových struktur obvyklých pro technicky zaměřené texty (například rovnice~(\ref{equation}) nebo Definice \ref{def:1} na straně \pageref{def:1}). Pro odkazovaní na vzorce a struktury zásadně používáme příkaz \verb \label  a~\verb \ref  případně \verb \pageref  pokud se chceme odkázat na stranu výskytu.

Na titulní straně je využito sázení nadpisu podle optického středu s využitím zlatého řezu. Tento postup byl probírán na přednášce. Dále je použito odřádkování se zadanou relativní velikostí 0.4 em a 0.3 em.
\section{Matematický text}
Nejprve se podíváme na sázení matematických symbolů a~výrazů v plynulém textu včetně sazby definic a vět s využitím balíku \texttt{amsthm}. Rovněž použijeme poznámku pod čarou s použitím příkazu \verb \footnote . Někdy je vhodné použít konstrukci \verb \mbox{} , která říká, že text nemá být zalomen.

\begin{definition}
\label{def:1}
\emph{Zásobníkový automat} (ZA) je definován jako sedmice tvaru 
      $A = (Q,\Sigma,\Gamma,\delta,q_0,Z_0,F)$, kde:
      \begin{itemize}
          \item Q je konečná množina \emph{vnitřních (řídicích) stavů},
          \item $\Sigma$ je konečná \emph{vstupní abeceda},
          \item $\Gamma$ je konečná \emph{zásobníková abeceda},
          \item $\delta$ je \emph{přechodová funkce} $Q \times (\Sigma \cup \{ \epsilon \} ) \times \Gamma \rightarrow 2^{Q \times \Gamma^{\ast}}$,
          \item $q_0 \in Q$ je \emph{počáteční stav}, $Z_0 \in \Gamma$ je \emph{startovací symbol zásobníku} a $F \subseteq Q$ je množina \emph{koncových stavů.}
      \end{itemize}
\end{definition}

Nechť $P = (Q,\Sigma,\Gamma,\delta,q_0,Z_0,F)$ je zásobníkový automat. \emph{Kofigurací} nazveme trojici $(q,w,\alpha) \in Q\times\Sigma^\ast\times\Gamma^\ast$, kde\,$q$\,je aktuální stav vnitřního řízení, $w$ je dosud nezpracovaná část vstupního řetězce a $\alpha = Z_{i_1}Z_{i_2}\dots Z_{i_k}$ je obsah zásobníku\footnote{$Z_{i_1}$ je vrchol zásobníku}.

\subsection{Podsekce obsahující větu a odkaz}
\begin{definition}
\label{def:2}
\emph{Řetězec\,$w$\,nad abecedou $\Sigma$ je přijat ZA} A jestliže $(q_{0},w,Z_{0}) \underset{A}{\overset{\ast}{\vdash}} (q_{F},\epsilon,\gamma)$ pro nějaké $\gamma \in \Gamma^{\ast}$ a $q_{F} \in F$. Množinu $L(A) = \{w \mid w$ je přijat $\mbox{ZA}\ A\}\subseteq \Sigma^{\ast}$ nazýváme \emph{jazyk přijímaný TS} $M$.
\end{definition}

Nyní si vyzkoušíme sazbu vět a důkazů opět s použitím balíku \texttt{amsthm}.

\begin{lemma}
Třída jazyků, které jsou přijímány ZA, odpovídá \emph{bezkontextovým jazykům}.
\end{lemma}
\begin{proof}
V důkaze vyjdeme z Definice \ref{def:1} a \ref{def:2}.
\end{proof}
%$(\frac{m}{n})$
\section{Rovnice a odkazy}
Složitější matematické formulace sázíme mimo plynulý text. Lze umístit několik výrazů na jeden řádek, ale pak je třeba tyto vhodně oddělit, například příkazem \verb \quad .

$$\sqrt[i]{x^{3}_{i}}\quad \text{kde } x_{i} \text{ je $i$-té sudé číslo splňující}\quad x^{2-x^{i^{2}}_{i}}_{i} \leq x^{y^{3}_{i}}_{i}$$

V rovnici (\ref{equation}) jsou využity tři typy závorek s různou explicitně definovanou velikostí.

\begin{eqnarray}
\label{equation}
x &= &\bigg[ \Big\{ \big[ a+b \big] \ast c \Big\} ^{d} \ominus 1 \bigg] ^{1/2}\\
y &= & \lim_{x \to \infty} \frac{\frac{1}{\log_{10} x}}{\sin^{2} x + \cos^{2} x}\nonumber
\end{eqnarray}

V této větě vidíme, jak vypadá implicitní vysázení limity $\mathrm{lim_{n\to\infty}}\,f(n)$ v normálním odstavci textu. Podobně je to i s dalšími symboly jako $\prod_{i=1}^n 2^i$ či $\bigcap_{A\in\mathcal{B}} A$. V~případě vzorců $\lim\limits_{n\to\infty}f(n)$ a $\prod\limits_{i=1}^n 2^i$ jsme si vynutili méně úspornou sazbu příkazem \verb \limits .

\begin{eqnarray}
\int_b^a g(x)\,\mathrm{d}x &= & -\int\limits_a^b f(x)\,\mathrm{d}x\\
\overline{\overline{A \land B}} &\Leftrightarrow &\overline{\overline{A} \vee \overline{B}}
\end{eqnarray}

\section{Matice}
Pro sázení matic se velmi často používá prostředí \texttt{array} a závorky (\verb \left , \verb \right ). 
$$
\left[
\begin{array}{ccc}
    &\widehat{\beta + \gamma} & \hat{\pi}\\
    \vec{a} &\overleftrightarrow{AC} &
\end{array}
\right] = 1 \Longleftrightarrow \mathbb{Q} = \mathbf{R}
$$
$$
\mathbf{A} = 
        \begin{array}{|cccc|}
        a_{11} &a_{12} &\dots  &a_{1n}\\
        a_{21} &a_{22} &\dots  &a_{2n}\\
        \vdots &\vdots &\ddots &\vdots\\
        a_{m1} &a_{m2} &\dots &a_{mn}
    \end{array} =
    \begin{array}{rl}
        t &u\\
        v &w
    \end{array} = tw - uv
$$

Prostředí \texttt{array} lze úspěšně využít i jinde.

$$
\binom{n}{k} = 
\left\{
\begin{array}{ll}
     0 &\text{ pro } k < 0 \text{ nebo } k >n\\
     \frac{n!}{k!(n-k)!} & \text{ pro } 0 \leq k \leq n
\end{array}
\right.
$$
\end{document}
